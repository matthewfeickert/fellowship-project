\section{Statistical Model}

\subsection{Introduction to Channels}

In particle physics searches the range of parameters of interest are broken into different regions for consideration.~\cite{Cranmer:2015nia,Baak:2014wma}
For example, the region of phase space where the signal for a search is most likely to be observed is referred to as a signal region (SR).
While regions of phase space which are signal free and the background processes are well modeled are referred to as control regions (CR).
In the parlance of \texttt{HistFactory} these regions are called ``channels.''~\cite{Cranmer:2012sba}\\

A channel consists of a statistical model of the signal and background components.
Each component, or ``sample'', is a p.d.f. for a specific process (signal or background).
\[
 f_i\parenth{x | \vec{\theta}}
\]
Each bin of the channel can then be modeled as a mixture model of the samples
\[
 \sum_i w_i f_i, \qquad \sum_i w_i = 1
\]
and each bin (with model and data) is Poisson distributed according to the data and model
\[
 \text{Poisson}\parenth{\text{bin data} \left|\frac{\mu S f_s + B_b f_b}{\mu S + B_b}\right.}
\]

\begin{figure}[htpb]
 \center%
 \includegraphics[width=0.3\linewidth]{example_channel.pdf}
 \caption{An example of a channel.
  The channel contains binned data (crosses) and a binned model (histogram).
  The model is a product over the bins where each bin is Poisson distributed according to the data given a mixture model of the individual p.d.f.s.}\label{fig:example_channel}
\end{figure}

\subsection{Model Template}

The statistical model of interest~\cite{Cranmer:2012sba} can be represented as the product over bins
\begin{equation}
 \mathcal{P}\parenth{n_{cb}, a_p | \phi_p, \alpha_p, \gamma_b} = \parenth{\,\prod_{c\, \in \text{channels}} \,\prod_{b \in \text{bins}} \text{Pois}\parenth{n_{cb} | \nu_{cb}}} \cdot G\parenth{L_0 | \lambda, \Delta_L} \cdot \prod_{p \in S + \Gamma} f_p \parenth{a_p | \alpha_p}
 \label{eq:model}
\end{equation}


\begin{equation}
 \nu_{cb}\parenth{\phi_p, \alpha_p, \gamma_b} = \lambda_{cs} \gamma_{cb}\, \phi_{cs}\parenth{\vec{\alpha}} \eta_{cs} \parenth{\vec{\alpha}} \sigma_{csb} \parenth{\vec{\alpha}}
 \label{eq:mean_number_events}
\end{equation}
